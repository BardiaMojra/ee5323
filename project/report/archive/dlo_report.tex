\documentclass[journal,twoside,web]{ieeecolor}
\usepackage{generic}
\usepackage{cite}
\usepackage{amsmath,amssymb,amsfonts}
\usepackage{algorithmic}
\usepackage{graphicx}
\usepackage{textcomp}
\graphicspath{{./images/}}
\usepackage{wrapfig}
\setlength{\parskip}{.35em}


\def\BibTeX{{\rm B\kern-.05em{\sc i\kern-.025em b}\kern-.08em
    T\kern-.1667em\lower.7ex\hbox{E}\kern-.125emX}}
\markboth{\journalname, VOL. XX, NO. XX, XXXX 2017}


% {Author \MakeLowercase{\textit{et al.}}: Preparation of Papers for IEEE TRANSACTIONS and JOURNALS (February 2017)}

% ref eq with \eqref{eq}



\title{State Dynamics and Shape Deformity Estimation and Tracking of Elastic Rods}

\author{Bardia Mojra\\}


\begin{document}

\maketitle





% \thanks{This paragraph of the first footnote will contain the date on
% which you submitted your paper for review. It will also contain support
% information, including sponsor and financial support acknowledgment. For
% example, ``This work was supported in part by the U.S. Department of
% Commerce under Grant BS123456.'' }
% \thanks{The next few paragraphs should contain
% the authors' current affiliations, including current address and e-mail. For
% example, F. A. Author is with the National Institute of Standards and
% Technology, Boulder, CO 80305 USA (e-mail: author@boulder.nist.gov). }
% \thanks{S. B. Author, Jr., was with Rice University, Houston, TX 77005 USA. He is
% now with the Department of Physics, Colorado State University, Fort Collins,
% CO 80523 USA (e-mail: author@lamar.colostate.edu).}
% \thanks{T. C. Author is with
% the Electrical Engineering Department, University of Colorado, Boulder, CO
% 80309 USA, on leave from the National Research Institute for Metals,
% Tsukuba, Japan (e-mail: author@nrim.go.jp).}}

\begin{abstract}


\end{abstract}

% \begin{IEEEkeywords}
% Enter key words or phrases in alphabetical
% order, separated by commas. For a list of suggested keywords, send a blank
% e-mail to keywords@ieee.org or visit \underline
% {http://www.ieee.org/organizations/pubs/ani\_prod/keywrd98.txt}
% \end{IEEEkeywords}

\section{Introduction}
\label{sec:introduction}
This is intro






\section{Conclusion}
A conclusion section


% \smallskip\noindent
% \begin{small}
% \begin{tabular}{l}
% \verb+\+\texttt{documentclass[journal,twoside,web]\{ieeecolor\}}\\
% \verb+\+\texttt{usepackage\{\textit{Journal\_Name}\}}
% \end{tabular}
% \end{small}

% \begin{figure}[!t]
% \centerline{\includegraphics[width=\columnwidth]{fig1.png}}
% \caption{Magnetization as a function of applied field.
% It is good practice to explain the significance of the figure in the caption.}
% \label{fig1}
% \end{figure}



% \begin{table}
% \caption{Units for Magnetic Properties}
% \label{table}
% \setlength{\tabcolsep}{3pt}
% \begin{tabular}{|p{25pt}|p{75pt}|p{115pt}|}
% \hline
% Symbol&
% Quantity&
% Conversion from Gaussian and \par CGS EMU to SI $^{\mathrm{a}}$ \\
% \hline
% $\Phi $&
% magnetic flux&
% 1 Mx $\to  10^{-8}$ Wb $= 10^{-8}$ V$\cdot $s \\
% $B$&
% magnetic flux density, \par magnetic induction&
% 1 G $\to  10^{-4}$ T $= 10^{-4}$ Wb/m$^{2}$ \\
% $H$&
% magnetic field strength&
% 1 Oe $\to  10^{3}/(4\pi )$ A/m \\
% $m$&
% magnetic moment&
% 1 erg/G $=$ 1 emu \par $\to 10^{-3}$ A$\cdot $m$^{2} = 10^{-3}$ J/T \\
% $M$&
% magnetization&
% 1 erg/(G$\cdot $cm$^{3}) =$ 1 emu/cm$^{3}$ \par $\to 10^{3}$ A/m \\
% 4$\pi M$&
% magnetization&
% 1 G $\to  10^{3}/(4\pi )$ A/m \\
% $\sigma $&
% specific magnetization&
% 1 erg/(G$\cdot $g) $=$ 1 emu/g $\to $ 1 A$\cdot $m$^{2}$/kg \\
% $j$&
% magnetic dipole \par moment&
% 1 erg/G $=$ 1 emu \par $\to 4\pi \times  10^{-10}$ Wb$\cdot $m \\
% $J$&
% magnetic polarization&
% 1 erg/(G$\cdot $cm$^{3}) =$ 1 emu/cm$^{3}$ \par $\to 4\pi \times  10^{-4}$ T \\
% $\chi , \kappa $&
% susceptibility&
% 1 $\to  4\pi $ \\
% $\chi_{\rho }$&
% mass susceptibility&
% 1 cm$^{3}$/g $\to  4\pi \times  10^{-3}$ m$^{3}$/kg \\
% $\mu $&
% permeability&
% 1 $\to  4\pi \times  10^{-7}$ H/m \par $= 4\pi \times  10^{-7}$ Wb/(A$\cdot $m) \\
% $\mu_{r}$&
% relative permeability&
% $\mu \to \mu_{r}$ \\
% $w, W$&
% energy density&
% 1 erg/cm$^{3} \to  10^{-1}$ J/m$^{3}$ \\
% $N, D$&
% demagnetizing factor&
% 1 $\to  1/(4\pi )$ \\
% \hline
% \multicolumn{3}{p{251pt}}{Vertical lines are optional in tables. Statements that serve as captions for
% the entire table do not need footnote letters. }\\
% \multicolumn{3}{p{251pt}}{$^{\mathrm{a}}$Gaussian units are the same as cg emu for magnetostatics; Mx
% $=$ maxwell, G $=$ gauss, Oe $=$ oersted; Wb $=$ weber, V $=$ volt, s $=$
% second, T $=$ tesla, m $=$ meter, A $=$ ampere, J $=$ joule, kg $=$
% kilogram, H $=$ henry.}
% \end{tabular}
% \label{tab1}
% \end{table}




% \newpage
% -------------------------------------------------------------
%% citation generator
% https://www.bibme.org/ieee-transactions-on-robotics
\newpage
\bibliography{ref}
\bibliographystyle{ieeecolor}
% -------------------------------------------------------------
\end{document}
